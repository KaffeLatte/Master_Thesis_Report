\section{Motivation}
Spotify today has more playlists than songs in their music library. Spotify also provides curated playlists as a form of music recommendation for their users. 

Given that a user has a preference for a specific playlist, an interesting feature would be to generate a playlist similar to the one a user has a preference for, but with different songs. This type of feature is interesting as it allows users to get music recommendations fitted to their needs. Such a feature could also give Spotify a competitive edge in the hardening competition for music streaming customers.

\section{What are Recommender Systems?}
Recommender systems provide an automated way to filter and rank information of interest for a certain user, possibly also taking time into account. A famous example of recommender systems is the product recommendation once initiated at Amazon, \textit{"Users who bought this product also bought"}. Another example of recommender systems are the movie recommendations provided by Netflix. Movie recommendations are interesting and non-trivial as a specific user at a certain time is likely to not be interested in the majority of movies provided by Netflix. The same thing applies to music, at any given moment a user is likely to not want to listen to the majority of songs in a music library . A last example of recommendation could be restaurant recommendation, where time and context are important factors. Recommending a simple hamburger restaurant is not likely to be of interest at date night, but it might be the perfect recommendation while driving the kids home after Saturday morning soccer game.

\section{What is Spotify?}
Spotify is a music streaming service which charges premium users a fee and presents free users with ads. Record companies are then paid according to the popularity of the tracks for which they hold digital rights. Spotify was launched in October 2008 and today has over 60 million active users, from which over 15 million are paying for the premium user service.

\section{Project Aim}
The aim of this thesis is to provide a scalable method for selecting candidate songs, in the context of playlist generation given a predefined playlist to mimic. This is an extension to the current field of music recommendation.  

\section{Limitations}
The work of this thesis is limited to finding candidate songs when generating playlists similar to Spotify \textit{Browse} playlists. The focus is on creating a scalable model of doing so. This means that any type of feature engineering is excluded from this thesis and the thesis is also limited from looking into the problem of ordering songs in playlist generation. 